\documentclass[a4paper]{ltjsarticle}

\usepackage{amsfonts} % 整数全体の集合とかの記号を表すフォント
\usepackage{bm}       % 数式でベクトルを表す太字斜体
\usepackage{braket}   % Dirac の Braket 記法

\usepackage{listings} % ソースコード
\usepackage{color}
\usepackage{comment}

%%%%%%%%%%%%%%%%%%%%%%%%%%%%%%%%%%%%%%%%%%%%%%%%%%%%%%%%%%%%%%%%%
%
% コード片
%
%%%%%%%%%%%%%%%%%%%%%%%%%%%%%%%%%%%%%%%%%%%%%%%%%%%%%%%%%%%%%%%%%

\begin{comment}
\lstset{ % モノクロ用
    basicstyle={\ttfamily},
    identifierstyle={},
    commentstyle={\small},
    keywordstyle={\bfseries},
    ndkeywordstyle={\small},
    stringstyle={\small\ttfamily},
    frame={tb},
    breaklines=true,
    columns=[l]{fullflexible},
    numbers=left,
    xrightmargin=0\zw,
    xleftmargin=3\zw,
    numberstyle={\scriptsize},
    stepnumber=1,
    numbersep=1\zw,
    lineskip=-0.5ex
}
\end{comment}

\lstset{
    basicstyle={\ttfamily},
    identifierstyle={},
    commentstyle={\color[rgb]{0,0.5,0}},
    keywordstyle={\bfseries \color[rgb]{0,0,1}},
    ndkeywordstyle={\small},
    stringstyle={\small\ttfamily \color[rgb]{1, 0.4, 0}},
    frame={tb},
    breaklines=true,
    columns=[l]{fullflexible},
    numbers=left,
    xrightmargin=0\zw,
    xleftmargin=3\zw,
    numberstyle={\scriptsize},
    stepnumber=1,
    numbersep=1\zw,
    lineskip=-0.5ex
}


\begin{document}
%%%% Title %%%%%%%%%%%%%%%%%%%%%%%%%%%%%%%%%%%%%%%%%%%%%%%%%%%%%%%%%%%%%%%%%%%%%
\title{コード片}
\author{Tsubasa}
\date{\today}
\maketitle
%%%% Content %%%%%%%%%%%%%%%%%%%%%%%%%%%%%%%%%%%%%%%%%%%%%%%%%%%%%%%%%%%%%%%%%%%

\section{\LaTeX でのコードの書き方}

% \lstinputlisting{test.c} ファイルを読み込む場合

{
\lstset{
    basicstyle={\ttfamily},
    identifierstyle={},
    commentstyle={\small},
    keywordstyle={\bfseries},
    ndkeywordstyle={\small},
    stringstyle={\small\ttfamily},
    frame={tb},
    breaklines=true,
    columns=[l]{fullflexible},
    numbers=left,
    xrightmargin=0\zw,
    xleftmargin=3\zw,
    numberstyle={\scriptsize},
    stepnumber=1,
    numbersep=1\zw,
    lineskip=-0.5ex
}
\begin{lstlisting}[caption=モノクロ, language=c]
#include <iostream>
int main() {
    // Comment ハローワールド
    std::out << "Hello World!\n";
    return 0;
}
\end{lstlisting}
}

\begin{lstlisting}[caption=カラー, language=c]
#include <iostream>
int main() {
    // Comment ハローワールド
    std::out << "Hello World!\n";
    return 0;
}
\end{lstlisting}

%%%%%%%%%%%%%%%%%%%%%%%%%%%%%%%%%%%%%%%%%%%%%%%%%%%%%%%%%%%%%%%%%%%%%%%%%%%%%%%%
\end{document}
