\documentclass[a4paper]{ltjsarticle}

%%%%%%%%%%%%%%%%%%%%%%%%%%%%%%%%%%%%%%%%%%%%%%%%%%%%%%%%%%%%%%%%%
%
% LaTeX と TikZ で作るおしゃれな見出し
% おしゃれな見出しを見つけるたびに追加予定
%
%%%%%%%%%%%%%%%%%%%%%%%%%%%%%%%%%%%%%%%%%%%%%%%%%%%%%%%%%%%%%%%%%

\usepackage{bxjalipsum}
\usepackage{color}
\usepackage{comment}
\usepackage{docmute}
\usepackage{fancyhdr} % ヘッダー
\usepackage[bottom=25mm]{geometry} % 余白等の設定
\usepackage{here}     % 図表をその場に表示
\usepackage[unicode]{hyperref} % PDF のメタデータ生成
\usepackage{listings} % コード
\usepackage{tcolorbox}
\tcbuselibrary{breakable,hooks,skins,xparse}
\usepackage{tikz}
\usetikzlibrary{calc,positioning,shapes}
\usepackage{titlesec} % Section などの設定
\usepackage{xparse}

\lstset{
    basicstyle={\ttfamily},
    identifierstyle={},
    commentstyle={\color[rgb]{0,0.5,0}},
    keywordstyle={\bfseries \color[rgb]{0,0,1}},
    ndkeywordstyle={\small},
    stringstyle={\small\ttfamily \color[rgb]{1, 0.4, 0}},
    frame={tb},
    breaklines=true,
    columns=[l]{fullflexible},
    numbers=left,
    xrightmargin=0\zw,
    xleftmargin=3\zw,
    numberstyle={\scriptsize},
    stepnumber=1,
    numbersep=1\zw,
    lineskip=-0.5ex
}

\begin{document}

\section*{はじめに}

\LaTeX と TikZ と tcolorbox で作ったおしゃれな見出しを紹介する。

TikZ や tcolorbox で見出しを作る場合、
section コマンド等を再定義することになる。
section コマンドはアスタリスクの有無で
見出しを目次に追加するかどうかを指定でき、
オプション引数で目次での見出しの内容、
通常の引数で目次の内容を指定できる。
よって section コマンドを再定義する場合、
上記のように引数を処理し目次に追加した後、
対応する章番号を1増やす処理が必要である。
そのような処理の例を次に示した。

\begin{lstlisting}[language=tex]
% \usepackage{xparse}
\RenewDocumentCommand\section{s o m}{
    \par
    \refstepcounter{section} % カウンタをインクリメント
    % スター付きでないなら目次に追加
    \IfBooleanF{#1}{
        \IfNovalueTF{#2}{
            \addcontentsline{toc}{section}{#3}
        }{
            \addcontentsline{toc}{section}{#2}
        }
    }
    % ここにセクション本体を書く
}
\end{lstlisting}

また、見出しは基本的にページの一番最後に来ないことが望ましい。
tcolorbox で作った box がページの最後に来ないようにするには
次のようにオプションを追加する。

\begin{lstlisting}[language=tex]
\begin{tcolorbox}[
    after={
        % ページの最後に来ることを回避
        \tcb@parfillskip@check
        \par\nointerlineskip
        \nobreak
        \vspace{1\zh} % ボックス後ろの余白
    },
]
% ここにボックスの中身を記述
\end{tcolorbox}
\end{lstlisting}

次のページから見出しの例を追加する。
例ではsectionコマンドを再定義するのではなく、
章番号や見出しの内容をハードコーディングした。
ソースコードは tex ファイルを参照。

\clearpage

% chapter
\noindent
\begin{tikzpicture}
    \node (n) [draw=black!60,line width=3pt,rounded corners=26pt,inner xsep=15pt,inner ysep=10pt]
        [anchor=north west] at (0, 0)
        [scale=1.3]
        {\headfont\Large 第 \raisebox{-5pt}{\HUGE 7} \Large 章};
    \node [fill=black!60,minimum width=\textwidth,minimum height=10pt,outer sep=0pt,rounded corners=5pt]
        [anchor=west] at (0, -70pt) {};
    \node [anchor=west,right=20pt of n.east] {\headfont\LARGE 吾輩は猫である};
\end{tikzpicture}

\vspace{\baselineskip}
%chapter end

\jalipsum[3]{wagahai}

% section
\vspace{\baselineskip}
\par\noindent
\tikz[baseline=(x.base)]{
    \node (x) [fill=black!70,rounded corners=8pt,inner xsep=6pt]
    {\textcolor{white}{\headfont\large 7.1}};
}
\hspace{1\zw}{\headfont\large セクション}
\vspace{.6\baselineskip}
\par
% end section

\jalipsum[3]{wagahai}

% subsection
\vspace{\baselineskip}
\par\noindent
{\headfont 7.1.1 \hspace{1em} サブセクション}
\par
% end subsection

\jalipsum[3]{wagahai}

\vspace{\baselineskip}
\hrule

\begin{itemize}
    \item デザイン参考:『はじめてのパターン認識』
    \item 実際は別のフォントを使用
\end{itemize}


%%%%%%%%%%%%%%%%%%%%%%%%%%%%%%%%%%%%%%%%%%%%%%%%%%%%%%%%%%%%%%%%%%%%%%%%%%%%%%%%
\end{document}
