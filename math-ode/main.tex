\documentclass[a4paper]{ltjsarticle}

\usepackage{amsfonts} % 整数全体の集合とかの記号を表すフォント
\usepackage{bm}       % 数式でベクトルを表す太字斜体
\usepackage{braket}   % Dirac の Braket 記法

\usepackage{mhchem}   % 化学式
\usepackage{chemfig}  % 構造式

\newcommand{\C}{\mathrm{C}\,}

%%%%%%%%%%%%%%%%%%%%%%%%%%%%%%%%%%%%%%%%%%%%%%%%%%%%%%%%%%%%%%%%%
%
% 数式の書き方など備忘録
%
%%%%%%%%%%%%%%%%%%%%%%%%%%%%%%%%%%%%%%%%%%%%%%%%%%%%%%%%%%%%%%%%%

\begin{document}
%%%% Title %%%%%%%%%%%%%%%%%%%%%%%%%%%%%%%%%%%%%%%%%%%%%%%%%%%%%%%%%%%%%%%%%%%%%
\title{数学1E ノート}
\author{Tsubasa}
\date{\today}
\maketitle
%%%% Content %%%%%%%%%%%%%%%%%%%%%%%%%%%%%%%%%%%%%%%%%%%%%%%%%%%%%%%%%%%%%%%%%%%

\section{常微分方程式 Ordinary Differential Equation}

\subsection{定数分離 \quad $ f(y)dy = g(x)dx $}

$$ \int f(y) \,\mathrm{d}y = \int g(x) \,\mathrm{d}x + \C $$

例題
$$ y'=x^2e^{2y} $$
$$ y = -\frac{1}{2} \log\left(-\frac{2}{3} x^2 + \C \right) $$


\subsection{定数変化法 \quad $ \mathrm{定数分離できる形} = p(x) $}

まず $ \mathrm{左辺} = 0 $ の解を $ f(x) $ として、
積分定数 $\C$ を $C(x)$ として、代入し特殊解を求める。
$$ y = C (\mathrm{同次式の解}) + \mathrm{特殊解} $$

例題
$$ xy' + y = x $$
$$ y = \frac{1}{2}x + \C x^{-1} $$


\subsection{同次線形 $ y'' + ay' + by  = 0 $}

$ y = \exp(\lambda x) $ を代入して特性方程式 $ \lambda ^2 + a \lambda + b = 0$ を解く

$$ y = \begin{cases}
    \mathrm{C}_1 \exp(\lambda_1 x) + \mathrm{C}_2 \exp(\lambda_2 x) \\
    \mathrm{C}_1 \exp(\lambda_0 x) + \mathrm{C}_2 \, x \exp(\lambda_0 x) & (特性方程式が重解の時)
\end{cases} $$

例題
$$ y''' - 5y'' + 6y' = 0 $$
$$ y = \mathrm{C}_1 + \mathrm{C}_2 \exp(2x) + \mathrm{C}_3 \exp(3x) $$


\subsection{線形非同次 \quad $ y'' + ay' + by = f(x) $}

$ (左辺) = 0 $ の解を $ y_1, y_2 $ と求める。特殊解を定数変化法か未定係数法
( $ y = ax + b $ や $ y = e^{kx} $ などと置いてみる)で特殊解 $y_0$ を求める。
$$ y = \mathrm{C}_1 \, y_1 + \mathrm{C}_2 \, y_2 + y_0 $$

例題
$$ y'' + 3y' + 2y = xe^x $$
$$ y = \mathrm{C}_1 \, e^{-x} + \mathrm{C}_2 \, e^{-2x} + \left( \frac{1}{6}x - \frac{5}{36} \right) e^2 $$



\subsection{1階連立 ODE}
$$ \begin{cases}
    y_1' = ay_1 + by_2 \\
    y_2' = cy_1 + dy_2
\end{cases} $$

$ \mathbf{y} = {^t}[y_1, y_2] $ とすると
$$ \mathbf{y}' = \begin{bmatrix} a & b \\ c & d \\ \end{bmatrix} \mathbf{y} $$

対角化して

$$ \mathbf{y} = \mathrm{C}_1 \, \mathbf{v}_1 \exp(\lambda_1 x) + \mathrm{C}_2 \, \mathbf{v}_2 \exp(\lambda_2 x) $$



%%%%%%%%%%%%%%%%%%%%%%%%%%%%%%%%%%%%%%%%%%%%%%%%%%%%%%%%%%%%%%%%%%%%%%%%%%%%%%%%
\section{ベクトル解析}

省略

%%%%%%%%%%%%%%%%%%%%%%%%%%%%%%%%%%%%%%%%%%%%%%%%%%%%%%%%%%%%%%%%%%%%%%%%%%%%%%%%
\section{変分法}

汎関数 $ I[f] = \int F(x, f', f) dx $ の最小値を極小値を求める。

オイラーラグランジュ方程式から求められる。
$$ \frac{\partial}{\partial f} - \frac{d}{dx}\frac{\partial F}{\partial f'} = 0 $$

$x$ が現れない $ F(f', f) $ のときは
$$ F - f' \frac{\partial F}{\partial f'} = \mathrm{const} $$

\subsection{ラグランジュの未定乗数法}

minimize $ f(x, y) $ subject to $ g(x, y) = 0 $

$$ F(x,  y, \lambda) = f(x, y) + \lambda g(x, y) $$

%%%%%%%%%%%%%%%%%%%%%%%%%%%%%%%%%%%%%%%%%%%%%%%%%%%%%%%%%%%%%%%%%%%%%%%%%%%%%%%%
\end{document}
