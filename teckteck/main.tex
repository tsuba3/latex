\documentclass[b5paper,10pt]{ltjsarticle}

%%%%%%%%%%%%%%%%%%%%%%%%%%%%%%%%%%%%%%%%%%%%%%%%%%%%%%%%%%%%%%%%%
%
% TECKTECK 企画カード(仮)
% 作成者: 宗形 翼 (45 期)
%
%%%%%%%%%%%%%%%%%%%%%%%%%%%%%%%%%%%%%%%%%%%%%%%%%%%%%%%%%%%%%%%%%

\usepackage{bxjalipsum} % 吾輩は猫である (Lorem ipsum)
\usepackage{calc}
\usepackage[skip=2pt]{caption} % 図とキャプションの余白を調整
\usepackage{comment}
\usepackage{docmute}
\usepackage{fancyhdr} % ヘッダー
\usepackage[top=15mm,bottom=15mm,left=15mm,right=15mm]{geometry} % 余白等の設定
\usepackage{here}     % 図表をその場に表示
\usepackage[unicode]{hyperref} % PDF のメタデータ生成
\usepackage{ifthen}
\usepackage{luacode}
\usepackage{luacolor}
\usepackage{lua-ul}
\usepackage{multicol}
\usepackage{tabularx}
\usepackage{tcolorbox}
\tcbuselibrary{breakable,hooks,skins,xparse}
\usepackage{tikz}
\usepackage{titlesec} % Section などの設定
\usepackage{varwidth}

\makeatletter

\newlength{\wtarget}
\newlength{\wactual}
\newcommand{\kintou}[2]{\kintouwidth{#1\zw}{#2}}
\newcommand{\kintouwidth}[2]{%
    \setlength{\wtarget}{#1}%
    \settowidth{\wactual}{#2}%
    \ifthenelse{\lengthtest{\wtarget < \wactual}}{%
        \setlength{\wtarget}{1pt * \real{\strip@pt\wtarget} / \real{\strip@pt\wactual}}%
        \scalebox{\strip@pt\wtarget}[1]{#2}%
    }{%
        \makebox[\wtarget][s]{#2}%
    }%
}

\newcommand{\placeholder}[3]{%
    \fbox{%
        \begin{minipage}[c][#2-2\fboxsep-2\fboxrule][c]{#1-2\fboxsep-2\fboxrule}
            \noindent\centering\Huge #3
        \end{minipage}
    }
}

%   ____      _
%  / ___|___ | | ___  _ __ ___
% | |   / _ \| |/ _ \| '__/ __|
% | |__| (_) | | (_) | |  \__ \
%  \____\___/|_|\___/|_|  |___/
%

% 45 期
\definecolor{primaryColor}{rgb}{0,0.25,0}
\definecolor{primaryTextColor}{rgb}{0,0.25,0}
\definecolor{highlightColor}{rgb}{0.7,0.9,0.7}

% 46 期
% \definecolor{primaryColor}{rgb}{0.48,0,0.1}
% \definecolor{primaryTextColor}{rgb}{0.48,0,0.1}
% \definecolor{highlightColor}{rgb}{0.95,0.8,0.9}

% モノクロ
% \definecolor{primaryColor}{cmyk}{0,0,0,1}
% \definecolor{primaryTextColor}{rgb}{0,0,0}
% \definecolor{highlightColor}{cmyk}{0,0,0,0.2}

%  _____ _ _   _
% |_   _(_) |_| | ___
%   | | | | __| |/ _ \
%   | | | | |_| |  __/
%   |_| |_|\__|_|\___|
%
\newcommand*{\docgroupname}[1]{\gdef\@docgroupname{#1}}
\newcommand*{\doctype}[1]{\gdef\@doctype{#1}}
\newcommand*{\docversion}[1]{\gdef\@docversion{#1}}
\newcommand*{\yoteibi}[1]{\gdef\@yoteibi{#1}}
\newcommand*{\yobibi}[1]{\gdef\@yobibi{#1}}
\newcommand*{\keyword}[1]{\@namedef{keyword/\arabic{keyword}}{#1}\stepcounter{keyword}}
\newcommand*{\kiseki}[5]{%
    \@namedef{kiseki/1/\arabic{kiseki}}{#1}
    \@namedef{kiseki/2/\arabic{kiseki}}{#2}
    \@namedef{kiseki/3/\arabic{kiseki}}{#3}
    \@namedef{kiseki/4/\arabic{kiseki}}{#4}
    \@namedef{kiseki/5/\arabic{kiseki}}{#5}
    \stepcounter{kiseki}
}
\docgroupname{TECKTECK}
\doctype{山行企画カード 2020}
\docversion{第 0 版}
\yoteibi{}
\yobibi{}

\newcounter{keyword}
\setcounter{keyword}{1}
\newcounter{kiseki}
\setcounter{kiseki}{1}

\newenvironment{kisekilist}{
    \par
    \centering
    \begin{tabular}{llllllll}
}{
    \end{tabular}
    \par
}

\newcommand*{\kisekitableline}[1]{%
    \quad\kintou{6}{\@nameuse{kiseki/1/#1}}
    &(&
    \kintou{6}{\@nameuse{kiseki/2/#1}} &
    \@nameuse{kiseki/3/#1}\quad 期
    &)&
    \quad TEL: &
    \@nameuse{kiseki/4/#1} &
    \quad \@nameuse{kiseki/5/#1}\quad \\
    \hline
}

\renewcommand{\maketitle}{%
    \medskip\noindent%
    \makebox[0pt][l]{\color{primaryTextColor}\@docgroupname}%
    \makebox[\textwidth][c]{\@doctype}%
    \makebox[0pt][r]{\@docversion}

    \begin{tcolorbox}[colback=white,colframe=primaryColor]
        \centering\Huge{\@title}
    \end{tcolorbox}

    \noindent
    \makebox[\textwidth][l]{%
        \ifthenelse{\equal{\@yoteibi}{}}{}{%
            期日:\quad \@yoteibi \qquad%
        }%
        \ifthenelse{\equal{\@yobibi}{}}{}{%
        予備日:\quad \@yobibi%
        }%
    }%
    \makebox[0pt][r]{%
        \addtocounter{keyword}{-1}
        \foreach \i in {1,...,\arabic{keyword}} {
            \fbox{\@nameuse{keyword/\i}}%
        }%
    }

    \ifthenelse{\value{kiseki}=1}{}{%
    \noindent 企責:
    \begin{kisekilist}
        \directlua{printkisekitable(\arabic{kiseki}-1)}
    \end{kisekilist}
    }
}

\begin{luacode*}
function printkisekitable(n)
    for i=1,n do
    tex.sprint(string.format("\\kisekitableline{%d}", i))
    end
end
\end{luacode*}

%  _   _                _ _ _
% | | | | ___  __ _  __| | (_)_ __   ___  ___
% | |_| |/ _ \/ _` |/ _` | | | '_ \ / _ \/ __|
% |  _  |  __/ (_| | (_| | | | | | |  __/\__ \
% |_| |_|\___|\__,_|\__,_|_|_|_| |_|\___||___/
%

% section
\renewcommand*{\section}[2][]{%
    \par
    \addpenalty\@secpenalty
    \refstepcounter{section}
    \ifthenelse{\equal{#1}{}}{
        \addcontentsline{toc}{section}{#2}
    }{
        \addcontentsline{toc}{section}{#1}
    }
    \vspace{\baselineskip}%
    \begin{tcolorbox}[%
        blanker,
        borderline west={5pt}{0pt}{primaryColor},
        borderline south={1pt}{0pt}{primaryColor},
        left=10pt,
        boxsep=5pt,
        after={
            % ページの最後に来ることを回避
            \tcb@parfillskip@check
            \par\nointerlineskip
            \nobreak
            \vspace{1\zh}
        },
    ]
        \headfont\color{primaryTextColor}\large#2
    \end{tcolorbox}
}

% susbsection
\titleformat{\subsection}{\headfont\color{primaryTextColor}}{◆◆}{8pt}{}
\titlespacing{\subsection}{0pt}{\baselineskip}{0pt}

%% subsubsection
\titleformat{\subsubsection}{\color{primaryTextColor}\headfont}{}{0pt}{}

%     _    _         _                  _
%    / \  | |__  ___| |_ _ __ __ _  ___| |_
%   / _ \ | '_ \/ __| __| '__/ _` |/ __| __|
%  / ___ \| |_) \__ \ |_| | | (_| | (__| |_
% /_/   \_\_.__/|___/\__|_|  \__,_|\___|\__|
%

% Original: https://github.com/Yasunari/ascolorbox/blob/master/ascolorbox.sty
% LICENSE: https://github.com/Yasunari/ascolorbox/blob/master/LICENSE
\DeclareTColorBox{ascolorbox4}{ o m O{3} O{}}%
{enhanced, colback=white, colframe=white,
attach boxed title to top left={xshift=1cm,yshift=-\tcboxedtitleheight/2}, fonttitle=\bfseries,varwidth boxed title=0.85\linewidth, coltitle=black, fonttitle=\gtfamily,
enlarge top by=2mm, enlarge bottom by=2mm, breakable, sharp corners,
boxed title style={colback=white,left=0mm,right=0mm},
borderline={.75pt}{#3pt}{black,dotted},
underlay unbroken={\draw[black,line width=.5pt]
       (title.east|-frame.north east)--([xshift=-#3*4pt]frame.north east)
       arc [start angle=180, end angle=270, radius=#3*4pt]
       -- ([yshift=#3*4pt]frame.south east)
       arc [start angle=90, end angle=180, radius=#3*4pt]
       -- ([xshift=#3*4pt]frame.south west)
       arc [start angle=0, end angle=90, radius=#3*4pt]
       -- ([yshift=-#3*4pt]frame.north west)
       arc [start angle=270, end angle=360, radius=#3*4pt]
       -- (frame.north west-|title.west) ;
       \filldraw[fill=gray,draw=gray]
       (frame.north east) -- +(0pt,-#3*2pt) -- +(-#3*2pt,-#3*2pt) -- +(-#3*2pt,0pt) -- cycle;
       \filldraw[fill=gray,draw=gray]
       (frame.south east) -- +(-#3*2pt,0pt) -- +(-#3*2pt,#3*2pt) -- +(0pt,#3*2pt) -- cycle;
       \filldraw[fill=gray,draw=gray]
       (frame.south west) -- +(0pt,#3*2pt) -- +(#3*2pt,#3*2pt) -- +(#3*2pt,0pt) -- cycle;
       \filldraw[fill=gray,draw=gray]
       (frame.north west) -- +(0pt,-#3*2pt) -- +(#3*2pt,-#3*2pt) -- +(#3*2pt,0pt) -- cycle;
       },
underlay first={\draw[black,line width=.5pt]
       (title.east|-frame.north east)--([xshift=-#3*4pt]frame.north east)
       arc [start angle=180, end angle=270, radius=#3*4pt]
       -- (frame.south east) ;
       \draw[black,line width=.5pt]
       (frame.south west)
       -- ([yshift=-#3*4pt]frame.north west)
       arc [start angle=270, end angle=360, radius=#3*4pt]
       -- (frame.north west-|title.west) ;
       \filldraw[fill=gray,draw=gray]
       (frame.north east) -- +(0pt,-#3*2pt) -- +(-#3*2pt,-#3*2pt) -- +(-#3*2pt,0pt) -- cycle;
       \filldraw[fill=gray,draw=gray]
       (frame.north west) -- +(0pt,-#3*2pt) -- +(#3*2pt,-#3*2pt) -- +(#3*2pt,0pt) -- cycle;
       },
underlay middle={\draw[black,line width=.5pt]
       (frame.north east)--(frame.south east) ;
       \draw[black,line width=.5pt]
       (frame.south west)--(frame.north west) ;
       },
underlay last={\draw[black,line width=.5pt]
       (frame.north east) -- ([yshift=#3*4pt]frame.south east)
       arc [start angle=90, end angle=180, radius=#3*4pt]
       -- ([xshift=#3*4pt]frame.south west)
       arc [start angle=0, end angle=90, radius=#3*4pt]
       -- (frame.north west);
              \filldraw[fill=gray,draw=gray]
       (frame.south east) -- +(-#3*2pt,0pt) -- +(-#3*2pt,#3*2pt) -- +(0pt,#3*2pt) -- cycle;
       \filldraw[fill=gray,draw=gray]
       (frame.south west) -- +(0pt,#3*2pt) -- +(#3*2pt,#3*2pt) -- +(#3*2pt,0pt) -- cycle;
       },
IfValueTF={#1}{title=【#2】〈#1〉}{title=【#2】},#4}

\renewenvironment{abstract}{
    \centering
    \begin{ascolorbox4}{\small 概要・山の見どころ}[2][width=\textwidth-6\zw]
    \small
}{
    \end{ascolorbox4}
}

%  _____                          _
% |  ___|__  _ __ _ __ ___   __ _| |_ ___
% | |_ / _ \| '__| '_ ` _ \ / _` | __/ __|
% |  _| (_) | |  | | | | | | (_| | |_\__ \
% |_|  \___/|_|  |_| |_| |_|\__,_|\__|___/
%

% emph: emphsize text
% \emph make text bold.
% \emph* make text highlighted.
\renewcommand*{\emph}{\@ifstar\@emphs\@emph}
\newcommand*{\@emphs}[1]{\highLight[highlightColor]{#1}}
\newcommand*{\@emph}[1]{\textbf{#1}}

\makeatother

\begin{document}
%  _____ _ _   _
% |_   _(_) |_| | ___
%   | | | | __| |/ _ \
%   | | | | |_| |  __/
%   |_| |_|\__|_|\___|
%
\title{富士山}
\yoteibi{7 月 29 日}
\yobibi{7 月 31 日}
\keyword{1泊2日}
\keyword{朝発}
\kiseki{五神 真}{東京大学}{57}{080-987-654}{Hardbank}
\kiseki{御茶ノ水 お茶子}{お茶の水女子大学}{57}{090-123-456}{dokomo}
\maketitle

%     _    _         _                  _
%    / \  | |__  ___| |_ _ __ __ _  ___| |_
%   / _ \ | '_ \/ __| __| '__/ _` |/ __| __|
%  / ___ \| |_) \__ \ |_| | | (_| | (__| |_
% /_/   \_\_.__/|___/\__|_|  \__,_|\___|\__|
%
\begin{abstract}
    日本一の高さを誇り世界遺産にも登録された富士山。
    日本人である以上一度は登っておきたいですよね。
    さらに今年は天皇陛下が登ったことで知られるプリンスルートを登ります。
    プリンスルートは富士宮ルートと御殿場ルートのいいとこ取りをしたルートです。
    登りは雄大な眺めを楽しめ、下りは大砂走りができます!
    下山後には富士山の見える温泉で疲れを癒します。
\end{abstract}

%     _        _   _               ____  _
%    / \   ___| |_(_) ___  _ __   |  _ \| | __ _ _ __
%   / _ \ / __| __| |/ _ \| '_ \  | |_) | |/ _` | '_ \
%  / ___ \ (__| |_| | (_) | | | | |  __/| | (_| | | | |
% /_/   \_\___|\__|_|\___/|_| |_| |_|   |_|\__,_|_| |_|
%
\section{行動予定}
\usetikzlibrary{positioning}
\begin{figure}[H]
\newcommand{\figwidth}{\textwidth - 10mm}
\newcommand{\lineheight}{1.5cm}
\newcommand{\belowtime}[3]{
    \node [below=3pt of #1.south west,anchor=north west,inner sep=0pt] {\scriptsize #2};
    \node [below=3pt of #1.south east,anchor=north east,inner sep=0pt] {\scriptsize #3};
}
\newcommand{\belowtimex}[3]{
    \node [below=3pt of #1.south west,anchor=north west,inner sep=0pt,xshift=-3pt] {\scriptsize #2};
    \node [below=3pt of #1.south east,anchor=north east,inner sep=0pt,xshift=3pt] {\scriptsize #3};
}
\newcommand{\belowtimecenter}[2]{
    \node [below=3pt of #1.south,anchor=north,inner sep=0pt] {\scriptsize #2};
}
\centering
\begin{tikzpicture}[]
    \node [draw, anchor=west] (a) {新宿駅};
    \belowtime{a}{6:30}{7:02}
    \node [draw] (b) [right=5.5cm of a] {新松田駅};
    \belowtime{b}{8:32}{}
    \node [draw] (c) [right=1cm of b] {松田駅};
    \belowtime{c}{}{8:37}
    \node at (\figwidth, 0) [] (d) [inner sep=0pt] {};
    \node [draw] (e) [below=\lineheight of a.west,anchor=west] {御殿場駅};
    \belowtime{e}{9:23}{9:30}
    \node [draw] (f) [right=1.2cm of e] {富士宮口五合目};
    \belowtime{f}{10:30}{11:00}
    \node [draw] (g) [right=1.2cm of f] {宝永第一火口};
    \belowtime{g}{11:45}{11:55}
    \node [draw] (h) [right=2.2cm of g] {宝永山山頂};
    \belowtime{h}{12:55}{13:30}
    \node (i) [below=\lineheight of d.west,anchor=west,inner sep=0pt] {};
    \node (jj) [below=\lineheight of e.west,inner sep=0pt] {};
    \node [draw] (j) [right=2cm of jj] {走り六合};
    \belowtime{j}{13:55}{14:05}
    \node [draw] (k) [right=3.5cm of j] {\makebox[3\zw][s]{七合目}};
    \belowtimex{k}{15:30}{15:40}
    \node [draw] (l) [right=2.5cm of k] {赤岩八号館};
    \belowtime{l}{15:40}{16:50}

    \draw [double] (a) to node[above,align=left]{小田急小田原線(小田原行き)} (b);
    \draw [-] (b) to node[above]{\footnotesize 4分} (c);
    \draw [double] (c) to node[above]{JR 御殿場線} (d);
    \draw [-] (e) to node[above]{\footnotesize 45分} (f);
    \draw [-] (f) to node[above]{\footnotesize 25分} (g);
    \draw [-] (g) to node[above]{\footnotesize 1時間15分} (h);
    \draw [-] (h) to node[above]{} (i);
    \draw [-] (jj) to node[above]{\footnotesize 25分} (j);
    \draw [-] (j) to node[above]{\footnotesize 75分 + 休10分} (k);
    \draw [-] (k) to node[above]{\footnotesize 1時間10分} (l);
\end{tikzpicture}
\end{figure}

\begin{table}[H]
\begin{tabular}{lll}
    \kintou{5}{歩程} &:& 4 時間 35 分 \\
    \kintou{5}{累積高低差} &:& UP: 1079 m / DOWN: 92 m \\
    \kintou{5}{標高} &:& 3300 m (赤岩八号館) \\
\end{tabular}
\end{table}

\clearpage

%  __  __
% |  \/  | __ _ _ __  ___
% | |\/| |/ _` | '_ \/ __|
% | |  | | (_| | |_) \__ \
% |_|  |_|\__,_| .__/|___/
%              |_|
\section{概念図}

\begin{figure}[H]
\begin{minipage}{0.48\textwidth}
    \placeholder{\textwidth}{0.5\textwidth}{A}
\end{minipage}
\hfill
\begin{minipage}{0.48\textwidth}
    \placeholder{\textwidth}{0.5\textwidth}{B}
\end{minipage}

\vspace{2ex}\centering\placeholder{0.9\textwidth}{\textwidth}{C}
\end{figure}

\clearpage

%   ____          _
%  / ___|___  ___| |_
% | |   / _ \/ __| __|
% | |__| (_) \__ \ |_
%  \____\___/|___/\__|
%
\section{費用}

\begin{table}[H]
    \begin{tabularx}{\textwidth}{llrlX}
        \kintou{3}{\textbf{回収金}} &:& 12,000 &円& 行きのタクシー代+宿泊費+保険料+入湯料*自炊費\\
        \kintou{3}{合計} &:& 12,345 &円& 解散までの費用 \\
        && 15,456 &円& 解散後、アフターに参加する \\
        && 13,456 &円& 解散後、アフターに参加しない \\

        \kintou{3}{交通費}&:&
        2,300 &円& 解散までの費用 \\
        && 2,800 &円& 解散後、アフターに参加する場合の新宿までの費用 \\
        && 3,300 &円& 解散後、アフターに参加しない場合の新宿までの費用 \\
        \kintou{3}{食費}&:& $\alpha$ &円& \\
        \kintou{3}{施設費}&:& 9,000 &円& 宿泊費、入湯料 \\
        \kintou{3}{その他}&:& 500 &円& 保険料 \\
    \end{tabularx}
\end{table}

\section{費用詳細}

\subsection{交通費}

\begin{table}[H]
    \begin{tabularx}{\textwidth}{lXl}
        行き
        & \textbullet 東京駅〜大阪駅 & 9,800 円 / 人 \\
        & \textbullet 大阪駅〜富士山五合目(タクシー) & 2,800 円 / 人 \\
        帰り
        & \textbullet 東京駅〜大阪駅 & 9,800 円 / 人 \\
        & \textbullet 大阪駅〜富士山五合目(タクシー) & 2,800 円 / 人 \\
    \end{tabularx}
\end{table}

\subsection{施設費}

\begin{table}[H]
    \begin{tabularx}{0.5\textwidth}{Xl}
        \textbullet 山小屋 & 3,800 円 / 人 \\
        \textbullet 温泉 & 2,800 円 / 人 \\
    \end{tabularx}
\end{table}

\section{保険の加入}

全員一括で加入

%  ____            _    _               _     _     _
% |  _ \ __ _  ___| | _(_)_ __   __ _  | |   (_)___| |_
% | |_) / _` |/ __| |/ / | '_ \ / _` | | |   | / __| __|
% |  __/ (_| | (__|   <| | | | | (_| | | |___| \__ \ |_
% |_|   \__,_|\___|_|\_\_|_| |_|\__, | |_____|_|___/\__|
%                               |___/
\section{持ち物}

\begin{multicols}{2}
    \begin{itemize}
        \item 回収金(12,000 円)
        \item エスケープ用のお金(8,000 円 程度)
        \item 保険証
        \item 学生証
        \item IC カード(4,000 円程度)
        \item スマホ
        \item スマホの充電バッテリー(ない場合充電器を持参すれば有料で借りられる)
        \item 水(2L 程度)
        \item 防寒着(フリースやユニクロのウルトラライトダウンなど)
        \item 体温調節しやすい服
        \item 登山用具
        \begin{itemize}
            \item ザック
            \item 山靴
            \item 替えの靴紐
            \item ヘッドライト
            \item 1/25000 地図(事前にコースを書いておく)
            \item 本紙
            \item マップケース(ジップロック などで可)
        \end{itemize}
        \item 雨具
        \begin{itemize}
            \item セパレ
            \item ザックカバー
            \item 1 日目の昼食
            \item 行動食(お菓子)
            \item 非常食(カロリーメイトなど)
            \item コンパス
            \item ティッシュ
            \item 汗拭きシート
            \item カイロ
            \item 軍手
            \item 帽子
        \end{itemize}
        \item 保健用具
        \begin{itemize}
            \item 絆創膏
            \item 常備薬
            \item 着替え
            \item 汗拭きタオル
            \item 温泉用タオル
            \item 日焼け止め
            \item ゴミ袋
        \end{itemize}
        \item あると便利
        \begin{itemize}
            \item アイマスク
            \item 耳栓
            \item マスク
        \end{itemize}
    \end{itemize}
\end{multicols}

\section{企責の持ち物}

\begin{multicols}{2}
    \begin{itemize}
        \item 保険セット×4
        \item ゲイター
        \item 携帯用酸素ボンベ
        \item コッヘル小×2
        \item カートリッジ×2
        \item ストーブ×2
        \item 新聞紙
        \item ウェッティー
    \end{itemize}
\end{multicols}

\section{食糧計画}

\begin{table}[H]
\begin{tabular}{lcc}
    \hline
    & 1 日目 & 2 日目 \\
    朝食 & & 山小屋 \\
    昼食 & 各自 & 各自 \\
    夕食 & 山小屋 & \\
    \hline
\end{tabular}
\end{table}

ご来光前にレモネードを作る。

%  _____                           _
% | ____|_  ____ _ _ __ ___  _ __ | | ___
% |  _| \ \/ / _` | '_ ` _ \| '_ \| |/ _ \
% | |___ >  < (_| | | | | | | |_) | |  __/
% |_____/_/\_\__,_|_| |_| |_| .__/|_|\___|
%                           |_|
%                                   of sections
\section[せくしょん]{セクション}
\emph{\jalipsum{jugemu}}
\emph*{\jalipsum{jugemu}}
\subsection{サブセクション}
\jalipsum[1]{wagahai}
\subsubsection{サブサブセクション}
\jalipsum[1]{wagahai}

%%%%%%%%%%%%%%%%%%%%%%%%%%%%%%%%%%%%%%%%%%%%%%%%%%%%%%%%%%%%%%%%%%%%%%%%%%%%%%%%
\end{document}
