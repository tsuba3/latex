\documentclass[b5paper,10pt]{ltjsarticle}

%%%%%%%%%%%%%%%%%%%%%%%%%%%%%%%%%%%%%%%%%%%%%%%%%%%%%%%%%%%%%%%%%
%
% TECKTECK 企画カード(仮)
% 作成者: 宗形 翼 (45 期)
%
%%%%%%%%%%%%%%%%%%%%%%%%%%%%%%%%%%%%%%%%%%%%%%%%%%%%%%%%%%%%%%%%%

\input{preumble.tex}

\begin{document}
%  _____ _ _   _
% |_   _(_) |_| | ___
%   | | | | __| |/ _ \
%   | | | | |_| |  __/
%   |_| |_|\__|_|\___|
%
\title{富士山}
\yoteibi{7 月 29 日}
\yobibi{7 月 31 日}
\keyword{1泊2日}
\keyword{朝発}
\kiseki{五神 真}{東京大学}{57}{03-1234-5678}{Hardbank}
\kiseki{御茶ノ水 お茶子}{お茶の水女子大学}{57}{090-1234-4567}{dokomo}
\zaikyo{田中 斎藤}{日本女子大学}{57}{070-3456-3456}{au}
\maketitle

%     _    _         _                  _
%    / \  | |__  ___| |_ _ __ __ _  ___| |_
%   / _ \ | '_ \/ __| __| '__/ _` |/ __| __|
%  / ___ \| |_) \__ \ |_| | | (_| | (__| |_
% /_/   \_\_.__/|___/\__|_|  \__,_|\___|\__|
%
\begin{abstract}
    日本一の高さを誇り世界遺産にも登録された富士山。
    日本人である以上一度は登っておきたいですよね。
    さらに今年は天皇陛下が登ったことで知られるプリンスルートを登ります。
    プリンスルートは富士宮ルートと御殿場ルートのいいとこ取りをしたルートです。
    登りは雄大な眺めを楽しめ、下りは大砂走りができます!
    下山後には富士山の見える温泉で疲れを癒します。
\end{abstract}

%     _        _   _               ____  _
%    / \   ___| |_(_) ___  _ __   |  _ \| | __ _ _ __
%   / _ \ / __| __| |/ _ \| '_ \  | |_) | |/ _` | '_ \
%  / ___ \ (__| |_| | (_) | | | | |  __/| | (_| | | | |
% /_/   \_\___|\__|_|\___/|_| |_| |_|   |_|\__,_|_| |_|
%
\section{行動予定}
\begin{figure}[H]
\centering
\begin{actionplan}
    \spot{新宿駅}{6:30}{7:02}
    \ride{小田急小田原線(小田原行き)}
    \spot{新松田駅}{8:32}{}
    \actionpath{\small 乗り換え}{}{dashed}
    \spot{松田駅}{}{8:37}
    \ride{JR 御殿場線}[下の注釈]
    \spot{御殿場駅}[\shop \wc]{9:23}{9:30}
    \walk{45分}
    \spot{富士宮口五合目}[\camera \wc]{10:30}{11:00}
    \walk{25分}
    \spot{宝永第一火口}{11:45}{11:55}
    \walk{1時間15分}
    \spot{宝永山山頂}[\food \camera]{12:55}{13:30}
    \walk{25分}
    \spot{走り六合}{13:55}{14:05}
    \walk{75分 + 休 10 分}
    \spot{七合目}{15:30}{15:40}
    \walk{1時間10分}
    \spot{赤岩八号館}[\wc \food \bed]{15:40}{16:50}
\end{actionplan}
\end{figure}

\begin{table}[H]
\begin{tabular}{lll}
    \kintou{5}{歩程} &:& 4 時間 35 分 \\
    \kintou{5}{累積高低差} &:& UP: 1079 m / DOWN: 92 m \\
    \kintou{5}{標高} &:& 3300 m (赤岩八号館) \\
\end{tabular}
\end{table}

\clearpage

%  __  __
% |  \/  | __ _ _ __  ___
% | |\/| |/ _` | '_ \/ __|
% | |  | | (_| | |_) \__ \
% |_|  |_|\__,_| .__/|___/
%              |_|
\section{概念図}

\begin{figure}[H]
\begin{minipage}{0.48\textwidth}
    \placeholder{\textwidth}{0.5\textwidth}{A}
\end{minipage}
\hfill
\begin{minipage}{0.48\textwidth}
    \placeholder{\textwidth}{0.5\textwidth}{B}
\end{minipage}

\vspace{2ex}\centering\placeholder{0.9\textwidth}{\textwidth}{C}
\end{figure}

\clearpage

%   ____          _
%  / ___|___  ___| |_
% | |   / _ \/ __| __|
% | |__| (_) \__ \ |_
%  \____\___/|___/\__|
%
\section{費用}

\begin{table}[H]
    \begin{tabularx}{\textwidth}{llrlX}
        \kintou{3}{\textbf{回収金}} &:& 12,000 &円& 行きのタクシー代+宿泊費+保険料+入湯料*自炊費\\
        \kintou{3}{合計} &:& 12,345 &円& 解散までの費用 \\
        && 15,456 &円& 解散後、アフターに参加する \\
        && 13,456 &円& 解散後、アフターに参加しない \\

        \kintou{3}{交通費}&:&
        2,300 &円& 解散までの費用 \\
        && 2,800 &円& 解散後、アフターに参加する場合の新宿までの費用 \\
        && 3,300 &円& 解散後、アフターに参加しない場合の新宿までの費用 \\
        \kintou{3}{食費}&:& $\alpha$ &円& \\
        \kintou{3}{施設費}&:& 9,000 &円& 宿泊費、入湯料 \\
        \kintou{3}{その他}&:& 500 &円& 保険料 \\
    \end{tabularx}
\end{table}

\section{費用詳細}

\subsection{交通費}

\begin{table}[H]
    \begin{tabularx}{\textwidth}{lXl}
        行き
        & \textbullet 東京駅〜大阪駅 & 9,800 円 / 人 \\
        & \textbullet 大阪駅〜富士山五合目(タクシー) & 2,800 円 / 人 \\
        帰り
        & \textbullet 東京駅〜大阪駅 & 9,800 円 / 人 \\
        & \textbullet 大阪駅〜富士山五合目(タクシー) & 2,800 円 / 人 \\
    \end{tabularx}
\end{table}

\subsection{施設費}

\begin{table}[H]
    \begin{tabularx}{0.5\textwidth}{Xl}
        \textbullet 山小屋 & 3,800 円 / 人 \\
        \textbullet 温泉 & 2,800 円 / 人 \\
    \end{tabularx}
\end{table}

\section{保険の加入}

全員一括で加入

%  ____            _    _               _     _     _
% |  _ \ __ _  ___| | _(_)_ __   __ _  | |   (_)___| |_
% | |_) / _` |/ __| |/ / | '_ \ / _` | | |   | / __| __|
% |  __/ (_| | (__|   <| | | | | (_| | | |___| \__ \ |_
% |_|   \__,_|\___|_|\_\_|_| |_|\__, | |_____|_|___/\__|
%                               |___/
\section{持ち物}

\begin{multicols}{2}
    \begin{checkbox}
        \item 回収金(12,000 円)
        \item エスケープ用のお金(8,000 円 程度)
        \item 保険証
        \item 学生証
        \item IC カード(4,000 円程度)
        \item スマホ
        \item スマホの充電バッテリー(ない場合充電器を持参すれば有料で借りられる)
        \item 水(2L 程度)
        \item 防寒着(フリースやユニクロのウルトラライトダウンなど)
        \item 体温調節しやすい服
        \item 登山用具
        \begin{checkbox}
            \item ザック
            \item 山靴
            \item 替えの靴紐
            \item ヘッドライト
            \item 1/25000 地図(事前にコースを書いておく)
            \item 本紙
            \item マップケース(ジップロック などで可)
        \end{checkbox}
        \item 雨具
        \begin{checkbox}
            \item セパレ
            \item ザックカバー
            \item 1 日目の昼食
            \item 行動食(お菓子)
            \item 非常食(カロリーメイトなど)
            \item コンパス
            \item ティッシュ
            \item 汗拭きシート
            \item カイロ
            \item 軍手
            \item 帽子
        \end{checkbox}
        \item 保健用具
        \begin{checkbox}
            \item 絆創膏
            \item 常備薬
            \item 着替え
            \item 汗拭きタオル
            \item 温泉用タオル
            \item 日焼け止め
            \item ゴミ袋
        \end{checkbox}
        \item あると便利
        \begin{checkbox}
            \item アイマスク
            \item 耳栓
            \item マスク
        \end{checkbox}
    \end{checkbox}
\end{multicols}

\section{企責の持ち物}

\begin{multicols}{2}
    \begin{checkbox}
        \item 保険セット×4
        \item ゲイター
        \item 携帯用酸素ボンベ
        \item コッヘル小×2
        \item カートリッジ×2
        \item ストーブ×2
        \item 新聞紙
        \item ウェッティー
    \end{checkbox}
\end{multicols}

\section{食糧計画}

\begin{table}[H]
\begin{tabular}{lcc}
    \hline
    & 1 日目 & 2 日目 \\
    朝食 & & 山小屋 \\
    昼食 & 各自 & 各自 \\
    夕食 & 山小屋 & \\
    \hline
\end{tabular}
\end{table}

ご来光前にレモネードを作る。

%  _____                           _
% | ____|_  ____ _ _ __ ___  _ __ | | ___
% |  _| \ \/ / _` | '_ ` _ \| '_ \| |/ _ \
% | |___ >  < (_| | | | | | | |_) | |  __/
% |_____/_/\_\__,_|_| |_| |_| .__/|_|\___|
%                           |_|
%                                   of sections
\section[せくしょん]{セクション}
\emph{\jalipsum{jugemu}}
\emph*{\jalipsum{jugemu}}
\subsection{サブセクション}
\jalipsum[1]{wagahai}
\subsubsection{サブサブセクション}
\jalipsum[1]{wagahai}

\begin{info}
    \jalipsum{jugemu}
\end{info}

夏目漱石

\begin{warn}
    \jalipsum[1]{wagahai}
\end{warn}

\jalipsum{jugemu}


%%%%%%%%%%%%%%%%%%%%%%%%%%%%%%%%%%%%%%%%%%%%%%%%%%%%%%%%%%%%%%%%%%%%%%%%%%%%%%%%
\end{document}
